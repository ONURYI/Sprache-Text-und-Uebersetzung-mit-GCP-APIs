\documentclass[12pt,a4paper]{article}
\usepackage[utf8]{inputenc}
\usepackage[ngerman]{babel}
\usepackage{graphicx}
\usepackage{booktabs} % Für Tabellen
\usepackage{caption}


\usepackage[hypertexnames=false]{hyperref} % Hinzugefügte Option

\begin{document}
	
		
		% Titelseite
		\begin{titlepage}
			\begin{center}
				\vspace*{1cm}
				\Huge
				Sprache, Spracheingabe, Text \& Übersetzung mit Google Cloud APIs
				
				
				\vspace{1.5cm}
				\LARGE
				M.Sc. Onur Yilmaz
				
				\vspace{1.5cm}
				\Large
				Angewandte Künstliche Intelligenz
				
				\vfill
				
				Schriftliche Ausarbeitung
				
				\vspace{0.8cm}
				\Large
				Betreuer: Prof. Dr. Giefers
				
				\vspace{0.5cm}
				\large
				Fachhochschule Südwestfalen
				\today
			\end{center}
		\end{titlepage}

\thispagestyle{empty}
\tableofcontents

\newpage
	\section{Einleitung} \cite{google}
	

\newpage
	\bibliographystyle{plain} % oder ein anderer Stil, der dir gefällt
	\bibliography{references} % Der Name deiner .bib-Datei ohne die Endung
	
\end{document}
