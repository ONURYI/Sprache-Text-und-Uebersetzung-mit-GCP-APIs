\documentclass[12pt,a4paper]{article}
\usepackage[utf8]{inputenc}
\usepackage[ngerman]{babel}
\usepackage{graphicx}
\usepackage{booktabs} % Für Tabellen
\usepackage{caption}
\usepackage{amsmath}
\usepackage{enumitem} % Paket für bessere Nummerierung
\usepackage{csquotes} 


\usepackage{listings}
\usepackage{xcolor}

% Einstellungen für das 'listings'-Paket
\lstset{
	basicstyle=\footnotesize\ttfamily,
	breaklines=true,
	frame=single,
	rulecolor=\color{gray},
	language=Python,
	keywordstyle=\color{blue},
	commentstyle=\color{gray},
	stringstyle=\color{red},
	numbers=left,
	numberstyle=\tiny\color{gray},
	captionpos=b,
	keepspaces=true,
	showstringspaces=false,
	extendedchars=true
}

 

\usepackage{tikz}
\usetikzlibrary{positioning}
\tikzstyle{block} = [rectangle, draw, fill=blue!20, text width=6em, text centered, rounded corners, minimum height=4em]
\tikzstyle{arrow} = [thick,-,>=stealth]


\usepackage{tikz}
\usepackage{hyperref} % Hinzugefügte Option

\usepackage[backend=biber, style=numeric]{biblatex}
\addbibresource{references.bib}
\begin{document}
	
		
		% Titelseite
		\begin{titlepage}
			\begin{center}
				\vspace*{1cm}
				\Huge
				Sprache, Spracheingabe, Text \& Übersetzung mit Google Cloud APIs
				
				
				\vspace{1.5cm}
				\LARGE
				M.Sc. Onur Yilmaz
				
				\vspace{1.5cm}
				\Large
				Angewandte Künstliche Intelligenz
				
				\vfill
				
				Schriftliche Ausarbeitung - Cloud Computing
				\vspace{0.5cm}
				\large
				Fachhochschule Südwestfalen
				\vspace{0.8cm}
				\Large
				\\
				Gutachter: Prof. Dr. Giefers
				\\
				\vspace{0.5cm}
				\large		
				\today
			\end{center}
		\end{titlepage}

\thispagestyle{empty}
\tableofcontents

\newpage
\section*{Einleitung}
In dieser Arbeit werden Technologien und Anwendungen aus dem Bereich der Sprach- und Textverarbeitung beleuchtet, die auf Google Cloud-Diensten aufbauen. Hierzu gehören die \textit{Cloud Speech API}, die \textit{Cloud Translation API}, die \textit{Natural Language API} und die \textit{Text-to-Speech API} \cite{googlecloudskills2023}.
\\ \\
Im Abschnitt über die \textit{Grundlagen} wird eine Einführung in die Konzepte des Cloud Computings, der API und des Umgangs mit dem Google Cloud Dienst gegeben. 
\\ \\
Der Abschnitt \textit{Spracherkennung und -transkription} fokussiert sich auf die \textit{Cloud Speech API}, die die Transkription von Audio in Text ermöglicht, und die Methoden zur Messung und Verbesserung der Sprachgenauigkeit.
\\ \\
In der \textit{Sprachübersetzung} wird die \textit{Cloud Translation API} behandelt, die den Prozess der Übersetzung von Texten in verschiedene Sprachen ermöglicht.
\\ \\
Der Bereich \textit{Textanalyse} befasst sich mit der \textit{Natural Language API}, die Techniken zur Klassifizierung von Text in Kategorien und zur Analyse von Entitäten und Sentiments bietet.
\\ \\
Im Abschnitt \textit{Sprachsynthese} wird die \textit{Text-to-Speech API} vorgestellt, die die Erzeugung synthetischer Sprache ermöglicht.
\\ \\
Die Arbeit dient nicht nur als theoretischer Überblick, sondern bietet auch praktische Einblicke und Anleitungen zur Verwendung dieser Tools. Dabei werden unterschiedliche Schwierigkeitsgrade und Themenbereiche abgedeckt, um einen umfassenden Einblick in die Möglichkeiten der Sprach- und Textverarbeitung mit Google Cloud zu bieten.

	
\newpage

\section{Grundlagen}
\subsection{Einführung in Cloud Computing}
Cloud Computing bezeichnet das Angebot und die Nutzung von IT Infrastrukturen und Anwendungen, die nicht lokal auf dem eigenen Computer, sondern in einem Netzwerk, meistens dem Internet, ausgeführt werden. Durch Cloud Computing können Unternehmen, aber auch Einzelpersonen Speicherplatz, Rechenleistung oder Softwareanwendungen als Service über das Internet nutzen.
\\ \\
Eine der Hauptanwendungen von Cloud Computing in der heutigen Zeit ist die Datenverarbeitung. Besonders hervorzuheben ist hier die Verwendung von Cloud-Plattformen für Natural Language Processing (NLP) Anwendungen, da sie die Fähigkeit bieten, große Mengen an Textdaten effizient zu analysieren und zu verarbeiten.

\subsection{Einführung in APIs}
Wenn man über Cloud spricht, ist eine der Hauptinteraktionen die Verwendung von APIs. Eine API (Application Programming Interface) dient als Schnittstelle, die es Entwicklern ermöglicht, bestimmte Funktionen eines Programms oder einer Plattform zu nutzen, ohne sich mit den internen Details auseinandersetzen zu müssen.

\subsubsection{Bedeutung von API-Keys}
\textit{API-Keys} sind ein wesentlicher Bestandteil vieler Webanwendungen und Cloud-Dienste. Sie dienen als Identifikator für den Benutzer oder Entwickler und stellen sicher, dass die Anfrage autorisiert ist. Durch den API-Key kann der Serviceanbieter den Zugriff auf seine Ressourcen kontrollieren und den Verbrauch überwachen.

\subsection{Erstellen und Aufrufen der API-Anfrage}
Das Aufrufen einer API, oft als \enquote{API-Anfrage} bezeichnet (\textit{im engl. request}), ist der Prozess, bei dem ein Programm oder eine Anwendung eine Anforderung an einen Server sendet und eine Antwort zurückerhält.

\begin{center}
	\begin{tikzpicture}
		% Anwendung
		\node[draw,rectangle,minimum width=3cm,minimum height=2cm] (app) {Anwendung};
		% API-Server
		\node[draw,rectangle,minimum width=3cm,minimum height=2cm, right=5cm of app] (api) {API-Server};
		
		% Pfeile für Anfrage und Antwort
		\draw[->, thick] (app.east) -- node[above]{Anfrage} (api.west);
		\draw[->, thick] (api.west) -- node[below]{Antwort} (app.east);
	\end{tikzpicture}
\end{center}

Im Kontext von NLP könnte eine API es ermöglichen, Textdaten zu senden und als Antwort eine Analyse oder Übersetzung dieses Textes zu erhalten.

\subsubsection{Grundlagen des API-Aufrufs}
Eine API-Anfrage ist eine Aufforderung an einen Server, eine bestimmte Aktion auszuführen oder Daten bereitzustellen. Um zu verstehen, wie eine solche Anfrage aufgebaut ist, betrachten wir ihre Hauptbestandteile:
\begin{enumerate}
	\item[i] \textbf{Endpoint (URL)}
	\item[ii] \textbf{HTTP-Methode}
	\item[iii]  \textbf{Header}
	\item[iv]  \textbf{Body}
\end{enumerate}

\subsubsection{Spezifikationen und Parameter}
Jeder Server oder Dienst kann mehrere Endpunkte haben, die unterschiedliche Arten von Daten oder Funktionen bedienen. Um die richtige Information oder Funktion zu erreichen, benötigen wir oft zusätzliche Details oder Parameter.

\begin{lstlisting}[language=Python, numbers = none]
	import requests
	
	url = "https://api.example.com/userinfo"
	headers = {
		"Authorization": "auth_token_here"
	}
	
	params = {
		"benutzerID": "12345"
	}
	
	response = requests.get(url, headers=headers, params=params)
	
	if  response.status_code == 200:
	benutzer_daten = response.json()
	print(benutzer_daten)
	else:
	print("Es gab ein Problem:", response.status_code)
\end{lstlisting}

\subsection{Von APIs zu RESTful APIs}
Innerhalb dieses Rahmens existiert eine spezielle Kategorie von APIs, die als \textit{RESTful APIs} bezeichnet werden. RESTful APIs nutzen das Internetprotokoll HTTP und nehmen seine Konventionen an, um eine klare Kommunikation zwischen Client und Server zu ermöglichen.

Ein signifikanter Vorteil von RESTful APIs ist ihre Zustandslosigkeit, die eine hohe Skalierbarkeit ermöglicht. 

Zum Beispiel:

\begin{itemize}
	\item Ein Aufruf von \texttt{getUsers()} in einer traditionellen API könnte in einer RESTful API als ein einfacher GET-Aufruf an die URL \texttt{/users} übersetzt werden.
	\item Das Hinzufügen eines neuen Benutzers, das in einer traditionellen API vielleicht durch \texttt{addUser()} repräsentiert wird, könnte in einer RESTful API durch eine POST-Anfrage an \texttt{/users} dargestellt werden.
\end{itemize}


















\subsection{Interaktion mit der Google Cloud API}
\subsubsection{Umgang mit dem Google Cloud Dienst}
\subsubsection{Erstellen eines API-Keys in Google Cloud}




\newpage
\section{Spracherkennung und -transkription}
\subsection{Schritte im Speech-to-Text Prozess}
\subsection{Google Cloud Speech API}


\newpage

\section{Sprachübersetzung}
\subsection{Erkennung und Übersetzung von Texten}
\subsection{Cloud Translation API}

\newpage

\section{Textanalyse}
\subsection{Klassifizierung von Text in Kategorien}
\subsection{Entitäten- und Sentimentanalyse}
\subsection{Natural Language API}

\newpage

\section{Sprachsynthese}
\subsection{Erzeugung synthetischer Sprache}
\subsection{Natural Language API}

\newpage
\thispagestyle{empty}
\printbibliography

\end{document}
