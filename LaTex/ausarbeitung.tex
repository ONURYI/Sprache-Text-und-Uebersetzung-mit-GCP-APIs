\documentclass[12pt,a4paper]{article}
\usepackage[utf8]{inputenc}
\usepackage[ngerman]{babel}
\usepackage{graphicx}
\usepackage{lipsum} % Für Beispieltext
\usepackage{booktabs} % Für Tabellen
\usepackage{caption}
\usepackage{hyperref}
\usepackage[backend=biber,style=numeric]{biblatex}

% Pfad zur .bib-Datei mit Literaturquellen
\addbibresource{references.bib}

\begin{document}
	
	% Titelseite
	\begin{titlepage}
		\begin{center}
			\vspace*{1cm}
			\Huge
			 Sprache, Spracheingabe, Text \& Übersetzung mit Google Cloud APIs
			
			
			\vspace{1.5cm}
			\LARGE
			M.Sc. Onur Yilmaz
			
			\vspace{1.5cm}
			\Large
			Angewandte Künstliche Intelligenz
			
			\vfill
			
			Schriftliche Ausarbeitung
			
			\vspace{0.8cm}
			\Large
			Betreuer: Prof. Dr. Giefers
			
			\vspace{0.5cm}
			\large
			Fachhochschule Südwestfalen
			\today
		\end{center}
	\end{titlepage}
	
	% Inhaltsverzeichnis
	\tableofcontents
	\newpage
	% Einleitung
	\section{Einleitung}
	\cite{google_cloud_speech_api}
	\section{Transkription von Sprache zu Text mit der Cloud Speech API}
	
	\section{Messung und Verbesserung der Sprachgenauigkeit}
	
	\section{Textübersetzung mit der Cloud Translation API}
	
	\section{Text in Kategorien einordnen mit der Natural Language API}
	
	\section{Entitäten- und Gefühlsanalyse mit der Natural Language API}
	
	\section{Synthetische Sprache erstellen mit Text-to-Speech}
	% Literaturverzeichnis
	\printbibliography[title={Literaturverzeichnis}]
	
\end{document}
