\documentclass[12pt,a4paper]{article}
\usepackage[utf8]{inputenc}
\usepackage[ngerman]{babel}
\usepackage{graphicx}
\usepackage{booktabs} % Für Tabellen
\usepackage{caption}
\usepackage{amsmath}
\usepackage{enumitem} % Paket für bessere Nummerierung
\usepackage{csquotes} 


\usepackage{listings}
\usepackage{xcolor}

% Einstellungen für das 'listings'-Paket
\lstset{
	basicstyle=\footnotesize\ttfamily,
	breaklines=true,
	frame=single,
	rulecolor=\color{gray},
	language=Python,
	keywordstyle=\color{blue},
	commentstyle=\color{gray},
	stringstyle=\color{red},
	numbers=left,
	numberstyle=\tiny\color{gray},
	captionpos=b,
	keepspaces=true,
	showstringspaces=false,
	extendedchars=true
}

 

\usepackage{tikz}
\usetikzlibrary{positioning}
\tikzstyle{block} = [rectangle, draw, fill=blue!20, text width=6em, text centered, rounded corners, minimum height=4em]
\tikzstyle{arrow} = [thick,-,>=stealth]


\usepackage{tikz}
\usepackage{hyperref} % Hinzugefügte Option

\usepackage[backend=biber, style=numeric]{biblatex}
\addbibresource{references.bib}
\begin{document}
	
		
		% Titelseite
		\begin{titlepage}
			\begin{center}
				\vspace*{1cm}
				\Huge
				Sprache, Spracheingabe, Text \& Übersetzung mit Google Cloud APIs
				
				
				\vspace{1.5cm}
				\LARGE
				M.Sc. Onur Yilmaz
				
				\vspace{1.5cm}
				\Large
				Angewandte Künstliche Intelligenz
				
				\vfill
				
				Schriftliche Ausarbeitung - Cloud Computing
				\vspace{0.5cm}
				\large
				Fachhochschule Südwestfalen
				\vspace{0.8cm}
				\Large
				\\
				Gutachter: Prof. Dr. Giefers
				\\
				\vspace{0.5cm}
				\large		
				\today
			\end{center}
		\end{titlepage}

\thispagestyle{empty}
\tableofcontents

\newpage
\section*{Einleitung}
In dieser Arbeit werden Technologien und Anwendungen aus dem Bereich der Sprach- und Textverarbeitung beleuchtet, die auf Google Cloud-Diensten aufbauen. Hierzu gehören die \textit{Cloud Speech API}, die \textit{Cloud Translation API}, die \textit{Natural Language API} und die \textit{Text-to-Speech API} \cite{googlecloudskills2023}.
\\ \\
Im ersten Abschnitt, der sich auf \cite{giefers2023cloud} bezieht, wird eine kurze Einführung in die Konzepte des Cloud Computings sowie der API-Integration im Zusammenhang mit NLP-Technologien gegeben, sowie die Interaktion mit der Google Cloud API erläutert.
\\ \\
Der Abschnitt \textit{Spracherkennung und -transkription} fokussiert sich auf die \textit{Cloud Speech API}, die die Transkription von Audio in Text ermöglicht, und die Methoden zur Messung und Verbesserung der Sprachgenauigkeit.
\\ \\
In der \textit{Sprachübersetzung} wird die \textit{Cloud Translation API} behandelt, die den Prozess der Übersetzung von Texten in verschiedene Sprachen ermöglicht.
\\ \\
Der Bereich \textit{Textanalyse} befasst sich mit der \textit{Natural Language API}, die Techniken zur Klassifizierung von Text in Kategorien und zur Analyse von Entitäten und Sentiments bietet.
\\ \\
Im Abschnitt \textit{Sprachsynthese} wird die \textit{Text-to-Speech API} vorgestellt, die die Erzeugung synthetischer Sprache ermöglicht.
\\ \\
Die Arbeit dient nicht nur als theoretischer Überblick, sondern bietet auch praktische Einblicke und Anleitungen zur Verwendung dieser Tools. Dabei werden unterschiedliche Schwierigkeitsgrade und Themenbereiche abgedeckt, um einen umfassenden Einblick in die Möglichkeiten der Sprach- und Textverarbeitung mit Google Cloud zu bieten.

	
\newpage

\section{Cloud Computing und NLP-Technologien}

\subsection{Cloud Computing im Kontext von NLP}
Cloud Computing bezeichnet die Bereitstellung von IT-Ressourcen wie Rechenleistung, Speicherplatz und Anwendungen über das Internet. Anstatt Ressourcen physisch vor Ort zu haben, ermöglicht Cloud Computing den Zugriff auf diese Ressourcen in großen Datenzentren. Dies bietet Unternehmen und Einzelpersonen Flexibilität, Skalierbarkeit und oft Kosteneffizienz. Im Kontext der natürlichen Sprachverarbeitung (NLP) revolutioniert Cloud Computing die Verarbeitung, Analyse und Interpretation großer Mengen von Sprachdaten. Es ermöglicht den Zugriff auf leistungsstarke und skalierbare Ressourcen, die NLP-Projekte effizient und effektiv durchführen.
\subsection{APIs und ihre Rolle im NLP}
Wenn man über Cloud spricht, ist eine der Hauptinteraktionen die Verwendung von APIs. 
\\ \\
Eine \textbf{\textit{API (Application Programming Interface)}} dient als Schnittstelle, die es Entwicklern ermöglicht, bestimmte Funktionen eines Programms oder einer Plattform zu nutzen, ohne sich mit den internen Details auseinandersetzen zu müssen. 
\\ \\
Das Aufrufen einer API, oft als \enquote{API-Anfrage} (\textit{im engl. request})  bezeichnet und ist der Prozess, bei dem ein Programm oder eine Anwendung eine Anforderung an einen Server sendet und eine Antwort zurückerhält.

\ \\ \\
\begin{tikzpicture}
	% Anwendung
	\node[draw,rectangle,fill=blue!20,minimum width=3cm,minimum height=2cm] (app) {Anwendung};
	% API-Server
	\node[draw,rectangle,fill=red!20,minimum width=3cm,minimum height=2cm, right=5cm of app] (api) {API-Server};
	
	% Pfeile für Anfrage und Antwort
	\draw[->, thick] (app.east) -- node[above]{Anfrage} (api.west);
	\draw[->, thick] (api.west) -- node[below]{Antwort} (app.east);
\end{tikzpicture}
\ \\ \\
Ein Großteil der modernen APIs, insbesondere im Cloud-Bereich, basiert auf dem Prinzip von REST (Representational State Transfer).\\ \\
\textit{\textbf{RESTful APIs}} nutzen Standard-HTTP-Methoden und bieten einen einheitlichen und vorhersehbaren Mechanismus, um mit dem Server zu interagieren. Dies erleichtert die Integration in Anwendungen und ermöglicht eine breitere Kompatibilität zwischen verschiedenen Systemen.
\\ \\
Im Kontext von NLP und Cloud Computing bieten RESTful APIs eine schnelle, zuverlässige und sichere Möglichkeit, auf leistungsstarke NLP-Modelle und -Dienste zuzugreifen. Sie ermöglichen es Entwicklern, Daten in Echtzeit zu verarbeiten und sofortige Analysen oder Antworten zu erhalten.
Zum Beispiel Textdaten zu senden und als Antwort eine Analyse oder Übersetzung dieses Textes zu erhalten, wäre eine beliebte und heutzutage häufig gebrauchte Anwendung ist. \\
\begin{center}
	\begin{tikzpicture}
		\node[draw, rectangle, fill=blue!20, minimum width=1cm, minimum height=1cm] (app) {NLP-Anwendung};
		\node[draw, rectangle, fill=red!20, minimum width=1.5cm, minimum height=1.5cm, right=of app, xshift=2cm] (api) {API};
		\node[draw, rectangle, fill=green!20, minimum width=1cm, minimum height=1cm, right=of api, xshift=2cm] (server) {NLP-Modell};
		
		\draw[->] (app) -- (api) node[midway, above] {Texteingabe};
		\draw[->] (api) -- (server) node[midway, above] {Verarbeitung};
		\draw[->] (server) -- (api) node[midway, below] {Antwort};
		\draw[->] (api) -- (app) node[midway, below] {Analyse};
	\end{tikzpicture}
\end{center}
\subsection{Interaktion mit der Google Cloud API}
Die Google Cloud Plattform bietet eine Vielzahl von RESTful APIs, die speziell für NLP-Aufgaben konzipiert sind. Diese APIs profitieren von der leistungsstarken Infrastruktur von Google, welche es Entwicklern ermöglicht, auf fortschrittliche NLP-Funktionen zuzugreifen. Für den Zugriff auf die Google Cloud APIs ist eine Authentifizierung erforderlich, die häufig über \textbf{\textit{API-Keys}} erfolgt. Sobald die Authentifizierung erfolgreich ist, können Entwickler Daten senden, diese verarbeiten lassen und die Ergebnisse für ihre spezifischen Anwendungen oder Analysen abrufen.


%%%%%%%%%%%%%%%%%%%%%%%%%%%%%%%%%%%%%%%%%%%%%%%%%%%%%%%%%%%%%%%%%%%%%%%%%%%%%%%%%%%%%%%%%%%%%%
\newpage

\section{Spracherkennung und Transkription}
In diesem Abschnitt behandeln wir den Prozess der Spracherkennung und Texttranskription, einschließlich der Nutzung der Google Cloud Speech API zur Konvertierung von gesprochener Sprache in Textform \cite{speechtotext2023}. Um hochwertige Transkriptionen zu erhalten, werden wir außerdem auf die Messung und Verbesserung der Genauigkeit bei der Spracherkennung eingehen. Hierbei stützen wir uns auf \cite{speechaccuracy2023}.
\subsection{Sprache-zu-Text-Prozess}
\subsection{Messung und Verbesserung der Sprachgenauigkeit}
\subsection{Cloud Speech API}




%%%%%%%%%%%%%%%%%%%%%%%%%%%%%%%%%%%%%%%%%%%%%%%%%%%%%%%%%%%%%%%%%%%%%%%%%%%%%%%%%%%%%%%%%%%%%%

\newpage
\section{Sprachübersetzung}

In diesem Abschnitt wird die Erkennung und Übersetzung von Texten behandelt, um verschiedene Sprachen zu übersetzen. Hierbei wird auf die Möglichkeiten der Cloud Translation API von Google Cloud eingegangen \cite{cloudtranslation2023}.
\subsection{Erkennung und Übersetzung von Texten}
\subsection{Cloud Translation API}
\newpage

\section{Textanalyse}
In diesem Abschnitts steht die Klassifizierung von Texten in verschiedene Kategorien im Vordergrund. Dabei wird erläutert, wie die Natural Language API von Google Cloud eingesetzt werden kann, um Texte einer gründlichen Analyse zu unterziehen und sie in entsprechende Kategorien einzuordnen \cite{classtext2023}. Doch die Möglichkeiten gehen noch weiter. 
\\ \\
Durch die Entitäten- und Sentimentanalyse gewinnen wir tiefere Einblicke in die Texte. Diese Analysetechniken ermöglichen es, bedeutende Entitäten innerhalb des Textes zu identifizieren sowie die generelle Stimmung und emotionale Tonalität des Textinhalts zu erfassen. Dieser Abschnitt beleuchtet die vielfältigen Anwendungsgebiete der Natural Language API von Google Cloud auf diesem Gebiet \cite{entitysentiment2023}.
\subsection{Klassifizierung von Text in Kategorien}
\subsection{Entitäten- und Sentimentanalyse}
\subsection{Natural Language API}


\newpage

\section{Sprachsynthese}
Die Erzeugung synthetischer Sprache mittels Text-to-Speech-Technologie wird in diesem Abschnitt behandelt. Dabei wird beschrieben, wie die Natural Language API von Google Cloud genutzt werden kann, um aus Texten künstliche Sprachausgaben zu erzeugen \cite{texttospeech2023}.
\subsection{Erzeugung synthetischer Sprache}
\subsection{Natural Language API}



\newpage
\thispagestyle{empty}
\printbibliography

\end{document}
