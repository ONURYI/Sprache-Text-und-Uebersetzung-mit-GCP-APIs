\documentclass[12pt,a4paper]{article}
\usepackage[utf8]{inputenc}
\usepackage[ngerman]{babel}
\usepackage{graphicx}
\usepackage{booktabs} % Für Tabellen
\usepackage{caption}
\usepackage{amsmath}
\usepackage{enumitem} % Paket für bessere Nummerierung
\usepackage{csquotes} 


\usepackage{tikz}
\usetikzlibrary{positioning}
\tikzstyle{block} = [rectangle, draw, fill=blue!20, text width=6em, text centered, rounded corners, minimum height=4em]
\tikzstyle{arrow} = [thick,-,>=stealth]



\usepackage[hypertexnames=false]{hyperref} % Hinzugefügte Option

\begin{document}
	
		
		% Titelseite
		\begin{titlepage}
			\begin{center}
				\vspace*{1cm}
				\Huge
				Sprache, Spracheingabe, Text \& Übersetzung mit Google Cloud APIs
				
				
				\vspace{1.5cm}
				\LARGE
				M.Sc. Onur Yilmaz
				
				\vspace{1.5cm}
				\Large
				Angewandte Künstliche Intelligenz
				
				\vfill
				
				Schriftliche Ausarbeitung - Cloud Computing
				\vspace{0.5cm}
				\large
				Fachhochschule Südwestfalen
				\vspace{0.8cm}
				\Large
				\\
				Gutachter: Prof. Dr. Giefers
				\\
				\vspace{0.5cm}
				\large		
				\today
			\end{center}
		\end{titlepage}

\thispagestyle{empty}
\tableofcontents

\newpage
\section*{Einleitung}
Im Rahmen dieser Arbeit werden verschiedene Technologien und Anwendungen im Bereich Sprache und Textverarbeitung vorgestellt, die auf den Diensten von Google Cloud basieren, einschließlich der \textit{Cloud Speech API}, der \textit{Cloud Translation API}, der \textit{Natural Language API} und der \textit{Text-to-Speech API} \cite{google}.
\\ \\
Im Abschnitt über die \textit{Grundlagen} wird eine Einführung in die Konzepte der API und des Umgangs mit dem Google Cloud Dienstes eingeganen.
\\ \\
Der Abschnitt \textit{Spracherkennung und -transkription} fokussiert sich auf die \textit{Cloud Speech API}, die die Transkription von Audio in Text ermöglicht, und die Methoden zur Messung und Verbesserung der Sprachgenauigkeit.
\\ \\
In der \textit{Sprachübersetzung} wird die \textit{Cloud Translation API} behandelt, die den Prozess der Übersetzung von Texten in verschiedene Sprachen ermöglicht.
\\ \\
Der Bereich \textit{Textanalyse} befasst sich mit der \textit{Natural Language API}, die Techniken zur Klassifizierung von Text in Kategorien und zur Analyse von Entitäten und Sentiments bietet.
\\ \\
Im Abschnitt \textit{Sprachsynthese} wird die \textit{Text-to-Speech API} vorgestellt, die die Erzeugung synthetischer Sprache ermöglicht.
\\ \\
Die Arbeit dient nicht nur als theoretischer Überblick, sondern bietet auch praktische Einblicke und Anleitungen zur Verwendung dieser Tools. Dabei werden unterschiedliche Schwierigkeitsgrade und Themenbereiche abgedeckt, um einen umfassenden Einblick in die Möglichkeiten der Sprach- und Textverarbeitung mit Google Cloud zu bieten.

	
\newpage

\section{Grundlagen}
\subsection{Was ist eine API}
\subsection{Erstellen eines API-Schlüssels}
\subsection{Erstellen und Aufrufen der API-Anfrage}
\subsection{Interaktion mit der Google Cloud API}
\subsection{Erstellen eines API-Keys in Google Cloud}


\newpage
\section{Spracherkennung und -transkription}
\subsection{Schritte im Speech-to-Text Prozess}
\subsection{Google Cloud Speech API}


\newpage

\section{Sprachübersetzung}
\subsection{Erkennung und Übersetzung von Texten}

\newpage

\section{Textanalyse}
\subsection{Klassifizierung von Text in Kategorien}
\subsection{Entitäten- und Sentimentanalyse}

\newpage

\section{Sprachsynthese}
\subsection{Erzeugung synthetischer Sprache}

\newpage
\bibliographystyle{plain} % oder ein anderer Stil, der dir gefällt
\bibliography{references} % Der Name deiner .bib-Datei ohne die Endung
	
\end{document}
